\chapter{KESIMPULAN DAN SARAN}


\section{Kesimpulan}

Berdasarkan hasil pengembangan dan implementasi sistem, dapat disimpulkan bahwa sistem backend yang dibuat untuk aplikasi PraktikaApp, PraktikaSSO, dan PraktikaTelelab berjalan dengan baik. Sistem ini telah berhasil di-deploy secara lokal dan dilengkapi dengan berbagai fitur keamanan yang mendukung, seperti \emph{OpenID Auth} untuk otentikasi pengguna. Selain itu, sistem backend ini dibangun dengan pendekatan \emph{microservice}, yang memungkinkan pengelolaan dan pengembangan aplikasi secara modular dan terpisah sesuai dengan fungsinya.

Selain itu, untuk memastikan kemudahan deployment dan skalabilitas, aplikasi ini dibungkus menggunakan Docker dan Kubernetes. Dengan demikian, sistem dapat dengan mudah di-deploy pada berbagai lingkungan dan mendukung proses pengelolaan kontainer yang efisien. Fitur keamanan lain yang diterapkan adalah Cross-Origin Resource Sharing (\emph{CORS}), yang memberikan perlindungan terhadap akses tidak sah dari sumber yang berbeda, memastikan aplikasi tetap aman dan terkontrol.

\section{Saran}

Sebagai pengembangan lebih lanjut, ada beberapa saran yang dapat dipertimbangkan untuk meningkatkan kualitas dan fungsionalitas sistem, antara lain:

\begin{itemize}
    \item Pengembangan aplikasi \textbf{Praktika App} dapat diperluas untuk mencakup berbagai komponen selain rangkaian digital, sehingga dapat mencakup berbagai aspek yang lebih luas dalam dunia pendidikan dan teknologi.
    \item Melakukan \emph{deployment} aplikasi di lingkungan \emph{production} yang saat ini belum dilakukan, agar aplikasi dapat digunakan oleh pengguna secara lebih luas dan memberikan feedback yang lebih komprehensif terkait performa dan kestabilan aplikasi di dunia nyata.
    \item Menambahkan fitur baru yang dapat memperkaya pengalaman pengguna, seperti analisis data penggunan aplikasi atau integrasi dengan layanan pihak ketiga untuk meningkatkan fungsionalitas sistem.
\end{itemize}
