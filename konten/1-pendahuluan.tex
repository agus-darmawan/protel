\chapter{PENDAHULUAN}
\sloppy
\section{Latar Belakang}
Pelaksanaan praktikum di Departemen Teknik Komputer ITS merupakan komponen penting dalam proses pembelajaran, di mana mahasiswa dapat mengaplikasikan teori yang telah dipelajari ke dalam praktik nyata. Namun, dalam pelaksanaannya, sejumlah kendala sering kali muncul yang menghambat kelancaran proses praktikum dan berdampak pada hasil belajar mahasiswa. Salah satu masalah utama yang dihadapi adalah kesulitan dalam mengakses informasi terkait praktikum. Saat ini, penyebaran informasi mengenai pendaftaran, pelaksanaan, teknis, modul, pengumpulan tugas, dan asistensi sering kali dilakukan melalui platform seperti WhatsApp. Meskipun WhatsApp merupakan alat komunikasi yang umum digunakan, penyampaian informasi melalui platform ini cenderung kurang terstruktur dan mudah terabaikan. Akibatnya, banyak mahasiswa yang terlambat atau bahkan tidak mendapatkan informasi penting yang berkaitan dengan pelaksanaan praktikum. Hal ini tentunya mengganggu persiapan dan partisipasi mahasiswa dalam praktikum, yang pada gilirannya mempengaruhi kualitas hasil belajar mereka.

Selain itu, kurangnya pemahaman mahasiswa terhadap materi yang disampaikan dalam praktikum juga menjadi tantangan yang signifikan. Meskipun modul panduan telah disediakan, mahasiswa masih sering mengalami kesulitan dalam memahami langkah-langkah praktikum yang harus dilakukan. Praktikum yang dilakukan secara berkelompok juga tidak jarang menyebabkan adanya ketidakmerataan dalam kontribusi antar anggota kelompok. Banyak mahasiswa yang hanya mengikuti instruksi dari rekan sekelompok tanpa terlibat secara aktif dalam pelaksanaan percobaan, sehingga pemahaman mereka terhadap materi menjadi kurang mendalam. Kebingungan yang terjadi selama praktikum juga disebabkan oleh minimnya simulasi atau prakondisi yang dapat memberikan gambaran awal sebelum praktik dimulai, yang pada akhirnya menghambat alur pelaksanaan praktikum. Minat dan antusiasme mahasiswa dalam mendalami materi praktikum juga terbilang rendah. Banyak mahasiswa yang lebih fokus pada penyelesaian laporan yang bersifat administratif, daripada benar-benar memahami dan menguasai materi yang dipraktikkan. Selain itu, kurangnya tantangan atau kompetisi yang dapat memotivasi mahasiswa juga menjadi faktor yang menurunkan minat mereka dalam mengikuti praktikum. Dengan latar belakang masalah-masalah tersebut, perlu adanya upaya untuk memperbaiki sistem pelaksanaan praktikum di Teknik Komputer ITS agar dapat memberikan pengalaman belajar yang lebih efektif dan mendalam bagi mahasiswa.

\section{Permasalahan}

Berdasarkan hal yang telah dipaparkan di latar belakang, terdapat beberapa masalah yang diangkat oleh peneliti dalam pembuatan tugas proyek telematika ini di antaranya:

\begin{enumerate}
\item \textbf{Kesulitan Akses Informasi Praktikum:} Informasi terkait pelaksanaan praktikum di Teknik Komputer ITS, seperti pendaftaran, teknis, modul, pengumpulan tugas, dan asistensi, saat ini disebarkan melalui WhatsApp. Meskipun WhatsApp merupakan alat komunikasi yang umum, penyampaian informasi melalui platform ini sering kali kurang terstruktur dan mudah terabaikan. Akibatnya, banyak mahasiswa yang terlambat mendapatkan informasi penting atau bahkan tidak menerima informasi sama sekali. Hal ini berdampak negatif pada persiapan dan partisipasi mahasiswa dalam praktikum, mengganggu alur kegiatan, dan menurunkan kualitas hasil pembelajaran.

\item \textbf{Kurangnya Pemahaman Materi Praktikum:} Meskipun modul panduan praktikum telah disediakan, banyak mahasiswa masih mengalami kesulitan dalam memahami materi yang disampaikan. Pelaksanaan praktikum yang dilakukan dalam kelompok sering kali tidak menjamin pemahaman yang merata di antara anggota kelompok. Banyak mahasiswa cenderung hanya mengikuti instruksi dari rekan sekelompok tanpa terlibat secara aktif dalam percobaan. Kondisi ini diperburuk oleh kurangnya simulasi atau prakondisi yang dapat memberikan gambaran awal sebelum praktikum dimulai, sehingga kebingungan sering terjadi selama proses pelaksanaan praktikum. Akibatnya, alur praktikum menjadi terhambat dan tujuan pembelajaran tidak tercapai secara optimal.
\end{enumerate}

\section{Solusi}

Untuk mengatasi masalah-masalah tersebut, solusinya adalah dengan mengembangkan sebuah sistem berbasis web yang mencakup berbagai fitur untuk memudahkan pelaksanaan praktikum. Solusi ini akan memberikan integrasi informasi, simulasi praktikum, serta sistem kompetisi yang mampu meningkatkan partisipasi dan pemahaman mahasiswa. Beberapa fitur utama dari sistem ini adalah:

\begin{enumerate}
\item \textbf{Sistem Administrasi Praktikum Berbasis Web:} Sistem ini akan memfasilitasi mahasiswa untuk mendaftar praktikum, memilih jadwal, dan mendapatkan notifikasi terkait informasi penting seperti modul, teknis pelaksanaan, hingga deadline pengumpulan. Sistem ini mirip dengan platform LARAS, namun dikembangkan khusus untuk kebutuhan praktikum Teknik Komputer. Dengan adanya notifikasi otomatis, mahasiswa tidak akan melewatkan informasi yang diperlukan.

\item \textbf{Praktikum Rangkaian Digital Berbasis Simulasi dan IoT:} Dalam praktikum rangkaian digital, solusi ini akan mencakup pretest berbasis simulasi rangkaian, yang dapat dilakukan sebelum praktikum dimulai. Selain itu, sistem akan terintegrasi dengan modul fisik melalui IoT, yang memungkinkan sistem mendeteksi apakah mahasiswa dapat menyelesaikan tugas modul atau tidak, secara real-time.
\end{enumerate}
